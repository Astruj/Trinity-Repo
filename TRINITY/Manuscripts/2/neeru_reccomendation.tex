\documentclass[12pt]{article}
\usepackage{url}
\usepackage[a4paper, total={6in, 8in}]{geometry}
\usepackage{hyperref}
\usepackage{titlesec}
\usepackage{graphicx}
\usepackage{amsmath}
\usepackage{amssymb}
\usepackage{mathtools}
\setcounter{secnumdepth}{4}
\usepackage{booktabs}
\usepackage{fancyhdr,graphicx,amsmath,amssymb}
\usepackage{algorithm,algpseudocode}
\usepackage{amsfonts}
\renewcommand{\algorithmicrequire}{\textbf{Input:}}
\renewcommand{\algorithmicensure}{\textbf{Output:}}

\include{pythonlisting}

\titleformat{\paragraph}
{\normalfont\normalsize\bfseries}{\theparagraph}{1em}{}
\titlespacing*{\paragraph}
{0pt}{3.25ex plus 1ex minus .2ex}{1.5ex plus .2ex}

\begin{document}

\title{Recommendation system for Wikipedia Gaze Based Personalized Summaries}
\maketitle

\begin{abstract}
  Due to it's complex collaborative structure and huge success, Wikipedia has been vastly analyzed from various perspectives. 
As a result, now we decently understand the overall nature of Wikipedia editors' collaboration dynamics and various features of it's articles. 
But a little research has been performed to understand readers' perspective of Wikipedia. 
In this paper, we propose a novel approach to analyze how a reader refers Wikipedia articles. 
This is attained by capturing the reading pattern of readers. 
We implement a state-of-the-art method to generate personalized summaries of Wikipedia articles through eye gaze tracking of a reader. 
These summaries capture reader's attention pattern. 
Summaries thus generated are gathered and analyzed for evaluation of different features of Wikipedia from readers' perspective. 
Using the proposed method, we develop a cross-platform document summarization and analysis tool. 
The experimental results show the efficiency of our personalized summary generation approach and the proposed analysis method of Wikipedia articles also show some interesting results.
\end{abstract}

\section{Introduction}

\subsection{Recommendation system}
papers: 
\begin{itemize}
\item The YouTube Video Recommendation System
\item Collaborative filtering and deep learning based recommendation system for cold start items
\item Automated web usage data mining and recommendation system using K-Nearest Neighbor (KNN) classification method
\item A personalized movie recommendation system based on collaborative filtering
\item Content-Based Video Recommendation System Based on Stylistic Visual Features
\item A Recommendation System Based on Hierarchical Clustering of an Article-Level Citation Network
\item Use of Deep Learning in Modern Recommendation System: A Summary of Recent Works
\item Science Concierge: A Fast Content-Based Recommendation System for Scientific Publications
\item Growing Wikipedia Across Languages via Recommendation
\item Evaluating Link-based Recommendations for Wikipedia
\end{itemize}

%%DUMMY
\cite{adler2007content} authors gain reputation when the edits and text
additions they perform to Wikipedia articles are longlived, and they lose reputation when their changes are

\section{Online Resources}\label{sec:Resources}


\section{Discussion}\label{sec:Discussion}


\section{Conclusion \& Future Work}\label{sec:Conclusion}

\newpage

\bibliography{neeru_reccomendation} 
\bibliographystyle{ieeetr}

\appendix

\section{Research Methods}

\subsection{Part One}



\subsection{Part Two}


\section{Online Resources}


\section{Words}
``averted" means tiled away

\end{document}